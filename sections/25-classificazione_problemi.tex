\section[Complessità Computazionale]{Classificazione dei problemi e complessità computazionale}
Abbiamo un problema e vogliamo progettare uno o più algoritmi per risolverlo.
Quando analizzo l'algoritmo e sono certo che funzioni vado a fare una stima delle risorse, solitamente tempo
e spazio, ma non solo.
Useremo il simbolo $\pi$ per riferirci ad un problema.\\
\begin{itemize}
\item \textbf{Limitazione superiore} (upper bound) f: $\mathbb{N} \rightarrow \mathbb{N}$\\
$f(n)$ risorsa $r$ è \emph{sufficiente} per risolvere $\pi$ se esiste un algoritmo $\mathcal{A}$
che risolve $\pi$ utilizzando su ogni input di lunghezza $n$ al più $f(n)$ risorsa $r$.\\
\item \textbf{Limitazione inferiore} (lower bound) g: $\mathbb{N} \rightarrow \mathbb{N}$\\
$g(n)$ risorsa $r$ è \emph{necessaria} per risolvere $\pi$ se per ogni algoritmo $\mathcal{A}$ che 
risolve $\pi$ esiste un input di lunghezza $n$ su cui $\mathcal{A}$ utilizza almeno $g(n)$ risorsa $r$.\\
\end{itemize}
\noindent Per fare un esempio, tra gli algoritmi di ordinamento $\Theta(n^2)$ rappresenta un upper bound (\texttt{insertionSort})
e $\Theta(n \log n)$ rappresenta un lower bound.


Un algoritmo è ottimale quando lower bound ed upper bound coincidono.
\subsection{Classi di complessità}
Siano\\
s,t : $\mathbb{N} \rightarrow \mathbb{N}$ due funzioni.\\
Una \emph{classe di complessità} è l'insieme dei problemi che possono essere risolti utilizzando la
"stessa" quantità di una determinata risorsa. (stessa è tra virgolette perchè non
ci basiamo su un valore preciso ma su una categoria più ampia).
Abbiamo innanzitutto la classe \textbf{P}, che è una classe di problemi 
$\pi$ che ammettono un algoritmo risolutivo che utilizza tempo polinomiale ($n^{O(1)}$).
Un esempio di un problema che non appartiene alla classe \textbf{P} è quello del commesso viaggiatore, ma
la maggior parte di quelli che abbiamo visto appartiene a questa classe.
Questo ci fa capire che una classe può contenere problemi di tipologie molto diverse.
Troviamo per esempio problemi di ricerca (albero ricoprente), problemi di ottimizzazione (albero ricoprente minimo),
e problemi di decisione. Un problema di decisione si può risolvere tramite un problema di ottimizzazione.
I problemi di decisione ci permettono di confrontare in maniera molto semplice problemi apparentemente diversi.
Spesso poi il problema di decisione ci permette di risolvere senza troppa fatica 
il problema di ottimizzazione associato.\\
Per esempio:
\begin{itemize}
    \item Ottimizzazione: dato un grafo trovare l'albero ricoprente minimo
    \item Decisione: dato un grafo, esiste l'albero ricoprente del grafo di peso $\le K$? 
\end{itemize}

\noindent Se $\pi$ è un problema di decisione e $\mathcal{A}$ è un algoritmo, $\mathcal{A}$ risolve $\pi$ quando
su input $x$ $\mathcal{A}$ restituisce 1 se e solo se $\pi(x) = 1$.\\
$\mathcal{A}$ risolve $\pi$ in tempo $t(n)$ e spazio $s(n)$ se e solo se 
$\mathcal{A}$ risolve $\pi$ utilizzando al più tempo $t(n)$ e al più spazio $s(n)$ su 
ogni input di lunghezza $n$.\\
Possiamo formalizzare questi concetti e vedere alcune classi di complessità:
\begin{itemize}
    \item \textbf{TIME}($t(n)$) = classe di problemi di decisione risolvibili in tempo $O(t(n))$
    \item \textbf{SPACE}($t(n)$) = classe di problemi di decisione risolvibili in spazio $O(s(n))$
    \item \textbf{CLASSE P} = $U_{c = 0}^{\infty}$ \textbf{TIME}($n^c$) (classe considerata risolvibile a tutti gli effetti)
    \item \textbf{PSPACE} = $U_{c = 0}^{\infty}$ \textbf{SPACE}($n^c$) spazio polinomiale
    \item \textbf{EXPTIME} = $U_{c = 0}^{\infty}$ \textbf{TIME}($2^{n^c}$) tempo esponenziale
\end{itemize}

\noindent Esistono alcune relazioni tra lo spazio utilizzato ed il tempo utilizzato da un algoritmo:
\begin{itemize}
    \item tempo polinomiale $\Rightarrow$ spazio polinomiale, quindi \textbf{P} $\subseteq$ \textbf{PSPACE}
    \item spazio polinomiale $\Rightarrow$ tempo esponenziale, quindi \textbf{PSPACE} $\subseteq$ \textbf{EXPTIME}
\end{itemize}

\noindent Da queste due considerazioni deduco che \textbf{P} $\subseteq$ \textbf{PSPACE} $\subseteq$ \textbf{EXPTIME}

\subsection{Problemi NP-completi}
Sono i problemi più difficili nella classe \textbf{NP} (problemi non deterministici in tempo polinomiale).
Vediamo alcuni esempi:
\subsubsection*{Problema delle partizioni}
Dato un insieme finito di oggetti, trovare due sottoinsiemi tali che la somma degli elementi dei due sottoinsiemi
sia uguale. Il tempo necessario per il calcolo delle partizioni è circa $2^n$ quindi ricade
in \textbf{EXPTIME}

\subsubsection*{Problema delle cricche}
Dato un grafo non orientato e un intero $k$, stabilire se esiste un sottografo completo
con $k$ vertici. La verifica avviene in tempo polinomiale, ma il costo della ricerca effettiva è
$\binom{n}{k}$ che può essere anche esponenziale.

\subsubsection{Problema soddisfacibilità (SODD)}
Istanza: Formula booleana $\Phi$ in forma normale congiuntiva con insieme di variabili $V$.\\
Questione: Esiste un assegnamento alle variabili in $V$ che rende vera $\Phi$, cioè tale che $\Phi(f) = 1$.\\
Per decidere se $\Phi$ è soddisfacibile proviamo tutti i possibili assegnamenti di valori alle variabili.
Quando si dice che la questione è vera e la clausola è soddisfacibile, bisogna portare il 
certificato, cioè la soluzione che risolve la formula.
Anche in questo caso verificare il problema è piuttosto semplice ma trovare la soluzione 
richiede tempo esponenziale.
Introduciamo il termine \emph{non deterministico}, ovvero un concetto utilizzato 
negli automi per definire la scelta di soluzione "indovinata".\\
\begin{algorithm}
    \caption{Problema di soddisfacibilità}
    \Indm\textbf{Algoritmo} \emph{sodd}($\Phi(x_1...x_n)$)\\
    \Indp\For{$i \leftarrow 1$ \textbf{to} $n$}{
        $z \leftarrow$ indovina un valore in $\lbrace 0,1 \rbrace$\\
    }
    $r \leftarrow$ valore della formula $\Phi(z_0...z_n)$ \texttt{/* verifica */}\\
    \Return{$r$}
\end{algorithm}\\
La verifica è polinomiale, ma cosa possiamo dire della computazione non deterministica di \emph{indovina}? 
Introduciamo la classe \textbf{NP} in cui abbiamo certificati verificabili 
in tempo polinomiale (ricordiamo che stiamo sempre parlando di problemi di decisione).\\
Chiamiamo \textbf{NTIME} di $f(n)$ la classe dei problemi che possono essere 
risolti da algoritmi non deterministici in tempo $V(f(n))$.\\
(\textbf{NP} sta per "Non Deterministic P" e \textbf{NON} "Non Polinomiale"!)\\
\textbf{NP} = $U^{\infty}_{c = 0}$ \textbf{NTIME}($n^c$)\\

\subsection{Relazioni tra classi di complessità}
Anche i problemi della partizione e della cricca sono risolvibili con un algoritmo \textbf{NP} in tempo $O(n)$ e $O(n^2)$, hanno le stesse caratteristiche
di SODD e quindi appartengono alla classe \textbf{NP}.\\
Lo stesso discorso fatto per \textbf{NSPACE} potremmo riprenderlo per \textbf{NPSPACE}.
\begin{center}
    Ogni algoritmo deterministico è un caso particolare di un algoritmo non deterministico
\end{center}
Possiamo quindi concludere che 
\begin{center}
    \textbf{P} $\subseteq$ \textbf{NP} $\subseteq$ \textbf{PSPACE} $\subseteq$ \textbf{EXPTIME}
\end{center}
Vogliamo ora dimostrare che \textbf{P} $\neq$ \textbf{NP}. Esiste qualcosa che sta in \textbf{NP} ma non in \textbf{P}?
Questi problemi sono detti NP-completi.\\
Supponiamo di avere due problemi $\pi_1 : I_1 \rightarrow \lbrace 0,1 \rbrace$ e $\pi_2 : I_2 \rightarrow \lbrace 0,1 \rbrace$.\\
$\pi_1$ è \emph{riconducibile} a $\pi_2$ se esiste f: $I_1 \rightarrow I_2$ tale che:
\begin{itemize}
    \item Per ogni $x \in I_1$ $\pi_1(x) = 1$ se e solo se $\pi_2(f(x)) = 1$
    \item $F$ è calcolabile in tempo polinomiale (nella lunghezza dell'input) da un algoritmo deterministico
\end{itemize}

La funzione $f$ è detta \emph{riduzione polinomiale}.\\
La riduzione trasforma un problema in un problema di un altro tipo, purchè 
se la risposta a $\pi_1$ è 1 anche la risposta a $\pi_2$ è 1, in tempo di soluzione polinomiale.\\
\textbf{Proprietà fondamentale}:
\begin{center}
    Se $\pi_1 \le \pi_2$ (è riduzione polinomiale) e $\pi_2 \in P$ allora $\pi_1 \in P$.
\end{center}

\noindent Ora possiamo definire formalmente i problemi NP-completi:
\begin{enumerate}
    \item $\pi$ problema di decisione è \textbf{NP-HARD} se per ogni $\pi' \in NP \rightarrow \pi' \le_p \pi$
    \item $\pi$ è NP-completo se è \textbf{NP-HARD} e $\in$ \textbf{NP} 
\end{enumerate}

\noindent Qualcuno è riuscito a dimostrare che SODD non è NP-completo.\\
Alcuni esempi di problemi NP-completi sono:
\begin{itemize}
    \item Cammino Hamiltoniano
    \item SODD3 (clausole grandi solo 3)
    \item Partizione
    \item Commesso viaggiatore
    \item Cammino massimo
\end{itemize}
\clearpage
