\section{Criterio di costo}
Suppongo di avere un algoritmo che trova il minimo in una sequenza di dati.

Se la sequenza è lunga {\emph{n}} elementi allora vengono fatti 
{\emph{n-1}} confronti ed un numero di assegnamenti compreso tra {\emph{n}} e {\emph{2n}}.\\
Assumendo che queste operazioni vengano effettuate in tempo costante il tempo è
$O(n)$, quindi posso dire che è $\Theta(n)$.\\

\subsection{Costo uniforme}
Ogni istruzione elementare utilizza un'unità di tempo indipendentemente 
dalla grandezza degli operandi.\newline

\noindent Ogni variabile elementare utilizza un'unità di spazio indipendentemente dal 
valore contenuto.

\subsection{Costo logaritmico}

Il tempo di calcolo di ciascuna operazione è proporzionale alla lunghezza dei
valori coinvolti.

\clearpage