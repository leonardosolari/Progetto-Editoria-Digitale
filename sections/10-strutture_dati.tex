\section{Strutture dati}
Le strutture dati consistono in una specifica organizzazione delle informazioni,
che permette di realizzare ed implementare un determinato tipo di dati.
La scelta della corretta struttura dati dipende dall'utilizzo che bisogna fare 
dei dati.

Il tipo di una variabile stabilisce i valori e le operazioni che possono essere
eseguite. In generale quando parliamo del tipo non parliamo della rappresentazione
del dato ma del "cosa". La rappresentazione influisce però sull'efficienza delle
operazioni.\\
Consideriamo un esempio classico: il dizionario. Si tratta di una collezione di elementi
ciascuno dei quali è caratterizzato da una chiave. Un esempio particolare di dizionario
può essere quello della lingua italiana in cui ogni elemento ha due campi, {\emph{parola}} e {\emph{definizione}},
oppure la registrazione di uno studente, in cui ogni elemento ha tanti campi e la chiave
è la {\emph{matricola}}.
Le chiavi in genere sono valori ordinabili.\\
In un dizionario dobbiamo poter svolgere le operazioni di {\textbf{ricerca}},
{\textbf{inserimento}} e {\textbf{cancellazione}}. \\
A seconda del tipo di struttura dati e di implementazione che si sceglie alcune operazioni possono essere più facili
da svolgere rispetto ad altre.
Vediamo ora alcune strutture dati.
\clearpage